\section{Drive}

\subsection{scaling}

At full output ({{ drive.denominator|default(1)|int }}) the axis will theoretically move at {{ drive.numerator|default(0)|float }}~{{vu}}.

\begin{itemize}
  \itemsep0em
  \item \itemIndent{Denominator}{ {{- drive.denominator|default(1)|int }}}
  \item \itemIndent{Numerator}{ {{- drive.numerator|default(0)|float }}}
\end{itemize}

\note{This is used for internal scaling only, it does not mean the axis can actually reach such velocities! \\
Consult the drive manual for details.
}

\subsection{links}

\subsubsection{core functions}

\begin{table}[H]
\centering
\caption{Core function links.}
\begin{tabularx}{\textwidth}{lX}
\hline
\rowcolor{Gray}
\textbf{function} & \textbf{link} \\
\hline
control & \verb|{{ drive.control }}| \\
status & \verb|{{ drive.status }}| \\
setpoint & \verb|{{ drive.setpoint }}| \\

reduced torque & \verb|{{ drive.control }}.{{ drive.reduceTorque|int }}| \\


drive reset & \verb|{{ drive.control }}.{{ drive.reset|int }}| \\


warning & \verb|{{ drive.status }}.{{ drive.warning|int }}| \\



error bit {{loop.index-1}} & \verb|{{drive.status }}.{{ bit|int }}| \\


\hline
\end{tabularx}
\label{tab:drive_core}
\end{table}



\subsubsection{input}

\begin{table}[H]
\centering
\caption{Input links}
\begin{tabularx}{\textwidth}{lX}
\hline
\rowcolor{Gray}
\textbf{function} & \textbf{link} \\
\hline
limit forward & \verb|{{ input.limit.forward }}| \\
limit backward & \verb|{{ input.limit.backward }}| \\
home & \verb|{{ input.home }}| \\
interlock & \verb|{{ input.interlock }}| \\
\hline
\end{tabularx}
\label{tab:drive_input}
\end{table}





\subsubsection{output}
\begin{table}[H]
\centering
\caption{Output links}
\begin{tabularx}{\textwidth}{lX}
\hline
\rowcolor{Gray}
\textbf{function} & \textbf{link} \\
\hline

brake & \verb|{{ output.brake|default('') }}| \\



health & \verb|{{ output.health|default('') }}| \\

\hline
\end{tabularx}
\label{tab:drive_output}
\end{table}





\note{brake not defined}

\subsection{brake}

\warning{You forgot to define the output for the brake! \\
Check the \texttt{output:}$\rightarrow$\texttt{brake:} section of the config file. \\
}

\note{Brake will open \emph{ {{-brake.openDelay|default(0)|int}} cycles} after the amplifier powered up. \\
The brake will engage \emph{ {{-brake.closeAhead|default(0)|int}} cycles} ahead of powering down the amplifier.}

\caution{Brakes are handled in EtherCAT cycles, rather than time!}

